\documentclass[11pt, oneside]{article}   	% use "amsart" instead of "article" for AMSLaTeX format
\usepackage{geometry}                		% See geometry.pdf to learn the layout options. There are lots.
\geometry{letterpaper}                   		% ... or a4paper or a5paper or ... 
%\geometry{landscape}                		% Activate for for rotated page geometry
%\usepackage[parfill]{parskip}    		% Activate to begin paragraphs with an empty line rather than an indent
\usepackage{graphicx}				% Use pdf, png, jpg, or eps§ with pdflatex; use eps in DVI mode
								% TeX will automatically convert eps --> pdf in pdflatex		
\usepackage{amssymb,amsmath}

\title{Calculating the Radial Component of the Tidal Field, I think}
\author{Daniel Martens}
%\date{}							% Activate to display a given date or no date

\begin{document}
\maketitle


We want to calculate:
\begin{equation}
S_{r r} = \hat{r}^{i} \hat{r}^{j} \nabla_{i} \nabla_{j} \nabla^{-2} \psi_{\alpha}
\end{equation}
In the basis where:
\begin{equation}
\psi_{\alpha} = j_{l}(k_{\alpha}r) Y_{l_{\alpha}m_{\alpha}}(\theta,\phi)
\end{equation}
Because we are in k-space, the inverse Laplacian operator is simple:
\begin{equation}
\nabla^{2}\psi_{\alpha} = -k_{\alpha}^{2}
$$
$$
\nabla^{-2}\psi_{\alpha} = \frac{-1}{k_{\alpha}^{2}}
\end{equation}

Before acting on the remaining terms with the directional gradients, it helps to first dot with the directional unit vectors:
\begin{equation}
\hat{r}^{i} \hat{r}^{j} \nabla_{i} \nabla_{j} = \hat{r}^{i} (  \nabla_{i} \hat{r}^{j} +\left[ \hat{r}^{j},\nabla_{i}\right] )\nabla_{j}
$$
$$
= \hat{r}^{i} \nabla_{i} \hat{r}^{j} \nabla_{j} - \hat{r}^{i} \nabla_{i} \hat{r}^{j} \nabla_{j} = \frac{d^{2}}{dr^{2}} - \hat{r}^{i}(\frac{1}{r}(\delta_{i}^{j} - \hat{r}^{j} \hat{r}_{i}))\nabla_{j}
$$
$$
= \frac{d^{2}}{dr^{2}}
\end{equation}


So we see that:
\begin{equation}
S_{rr} = \frac{-1}{k_{\alpha}^{2}} Y_{l_{\alpha}m_{\alpha}}(\theta,\phi) \frac{d^{2}}{dr^{2}} \left[j_{l}(k_{\alpha}r)  \right]
\end{equation}
Using the recursion relation of the spherical bessel functions of the first kind:
\begin{equation}
2 \frac{dj_{l}(x)}{dx} = j_{l-1}(x) - j_{l+1}(x)
\end{equation}
And the chain rule:
\begin{equation}
\frac{d}{dr}j_{l}(k_{\alpha}r) = k_{\alpha} \frac{d}{d(k_{\alpha}r)} j_{l}(k_{\alpha}r)
\end{equation}
We end with our result:
\begin{equation}
S_{r r} = \frac{Y_{l_{\alpha}m_{\alpha}}(\theta,\phi)}{4}\left[ 2 j_{l}(k_{\alpha}r) - j_{l+2}(k_{\alpha}r) - j_{l-2}(k_{\alpha}r) \right]
\end{equation}
















\end{document}  