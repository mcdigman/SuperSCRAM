%\documentclass[useAMS,usenatbib]{mn2e}

% uncomment for double space paper
\documentclass[11pt]{article}   			% use "amsart" instead of "article" for AMSLaTeX format
%\usepackage{geometry}                		% See geometry.pdf to learn the layout options. There are lots.
%\geometry{letterpaper}                   		% ... or a4paper or a5paper or ... 
%\usepackage{setspace}
%\doublespacing 


\usepackage{graphicx}
\usepackage{natbib}


\usepackage{amsmath, amssymb}
\usepackage{color}



\def\lesssim{\mathrel{\hbox{\rlap{\hbox{\lower5pt\hbox{$\sim$}}}\hbox{$<$}}}}
\def\gtrsim{\mathrel{\hbox{\rlap{\hbox{\lower5pt\hbox{$\sim$}}}\hbox{$>$}}}}
\def\mnras{MNRAS}
\def\apj{ApJ}
\def\apjl{ApJL}
\def\aj{AJ}
\def\planss{Planet.~Space Sci.}
\def\prd{Phys.~Rev.~D}
\def\apjs{ApJS}
\def\aap{A\&A}
\def\nat{Nature}
\def\apss{Ap\&SS}
\def\lcdm{$\Lambda$CDM}
\def\hMpc{h^{-1} \text{Mpc}}
\def\rgm{r_{\rm gm}}
\def\hMsun{h^{-1}M_\odot}

\def\xigg{\xi_{\rm gg}}
\def\xigm{\xi_{\rm gm}}
\def\ximm{\xi_{\rm mm}}
\def\Omegam{\Omega_{\rm m}}


\voffset=-0.69in
\hoffset=0.12in

\title[]{Super Sample Covariance Mitigation}
%%%%%%%%%%%%%%%%%%%%AUTHORS%%%%%%%%%%%%%%%%%%%%%%%%%%%%
\author[McEwen et al.]{ 
\parbox{\textwidth}{
Joseph E. McEwen$^{1,3}$\thanks{E-mail: mcewen.24@osu.edu},
Xiao Fang$^{1,3}$, 
Christopher M. Hirata$^{1,2,3}$}3
\\
$^1$ Department of Physics, The Ohio State University, Columbus, Ohio 43210, USA\\
$^2$ Department of Astronomy, The Ohio State University, Columbus, Ohio
43210, USA\\
$^3$ Center for Cosmology and Astro-Particle Physics, The Ohio State
University, Columbus, Ohio 43210, USA\\
}


\begin{document}

\maketitle

\begin{abstract}
Due to the finite survey volume of any large-scale survey, the super-sample covariance (SSC) is inevitably introduced by coupling between matter density modes within and beyond the survey scale, which contributes to the errors in statistical observables we measure. In this paper, a detailed quantification of SSC is carried out and our mitigation strategies are introduced. Our formalism has several highlighted advantages: (i) The effects of the survey geometry are considered and we are able to handle multiple surveys with different/overlapping geometries, which is essential for combining data from future experiments (DESI, PFS, LSST, WFIRST, Euclid, ...). (ii) The formalism can be applied to various observables, such as the weak lensing signal, TSZ effect and cluster counts. (iii) This mitigation formalism will be useful for optimizing survey design.
 \end{abstract}
 
 \section{Introduction}
Future and current surveys will seek to extract cosmological parameters from multiple observations. For instance the combination of galaxy-galaxy lensing and galaxy clustering has been used to constrain the cosmological parameters $\Omega_m$ (the amount of matter in the universe) and $\sigma_8$ (the parameter that sets the amplitude of the power spectrum). The information content of any such analysis depends on the associated errors in the model used. These errors are encapsulated in the covariance matrix. Hence, accurate covariance matrices are at the forefront of the cosmological ``to do list." In the linear regime the covariance matrix is Gaussian and simple to calculate. However, non-linear gravitational evolution will lead to non-Gaussian covariance matrix. A particular contaminant is the ``super sample covariance"; the covariance due to modes larger than the survey region, that are coupled to modes within the survey region by non-linear evolution. In this work we present a method to measure the super sample covariance by increasing the Fisher information content. To increase the Fisher information, we target specific observables that probe the super survey mode, for instance the number density of objects or the mean tangential shear around a survey boundary. When observations are independent Fisher matrices can be added. This procedure is schematically visualized as increasing the Fisher information. The covariance reduction is due to the fact that the covariance matrix is the inverse of the Fisher matrix.  

 
 
 
 
%\bibliographystyle{mn2e}
%\bibliography{sac}

\end{document}
