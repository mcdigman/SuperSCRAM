\documentclass[11pt, oneside]{article}   	% use "amsart" instead of "article" for AMSLaTeX format
\usepackage{geometry}                		% See geometry.pdf to learn the layout options. There are lots.
\geometry{letterpaper}                   		% ... or a4paper or a5paper or ... 
%\geometry{landscape}                		% Activate for for rotated page geometry
%\usepackage[parfill]{parskip}    		% Activate to begin paragraphs with an empty line rather than an indent
\usepackage{graphicx}				% Use pdf, png, jpg, or eps§ with pdflatex; use eps in DVI mode
								% TeX will automatically convert eps --> pdf in pdflatex		
\usepackage{amssymb,amsmath}

\newcommand{\sph}[2]{Y^\text{R}_{l_#1 m_#1}(\hat{#2})}
\newcommand{\jl}[1]{j_{l_#1}}
\newcommand{\dk}{\frac{ d^3 \mathbf{k}}{(2 \pi)^3}} 
\newcommand{ \dkv}[1]{\frac{ d^3 \mathbf{k}_{#1}}{(2 \pi)^3}} 
\newcommand{\obs}{\mathcal{O}}

\title{The $\delta_\alpha$ basis and covariance $\langle \delta_\alpha \delta_\beta \rangle$}
\author{Joseph E. McEwen}
%\date{}							% Activate to display a given date or no date

\begin{document}
\maketitle

\section{Spherical basis functions}
For spherical geometry we use the following basis, composed of the spherical Bessel function and spherical harmonic:
\begin{align} 
\psi_\alpha(r, \theta, \phi) = j_{l_\alpha}(k_\alpha r) Y^\text{R}_{l_\alpha m_\alpha}(\theta, \phi) ~, 
\end{align} 
where $Y^\text{R}_{lm}(\theta, \phi) $ is the real spherical harmonic. The real spherical harmonics are defined as 

\begin{align}
Y^\text{R}_{lm} =
\begin{cases} \frac{i}{\sqrt{2}} \left( Y_{lm}  - (-1)^m Y_{l-1m} \right)& \text{if }  m < 0 \\
Y_{l0} & \text{if } m=0 \\
\frac{1}{\sqrt{2}}\left( Y_{l-m} + (-1)^mY_{lm} & \text{if } m > 0 ~.
\end{cases}
\end{align} 
We write a super survey mode as 
\begin{align}
\delta_\alpha(k_\alpha)& = \int \delta(\mathbf{r}) \jl{\alpha}(k_\alpha r) \sph{\alpha}{r} d^3 \mathbf{r}~. 
\end{align}

\section{Mean background density in a region}
The mean background density in a region is composed of the super modes:

\begin{align}
\label{bar_delta}
\bar{\delta}= \displaystyle \sum_\alpha \frac{3}{r_\text{max}^3 - r_\text{min}^3} \int_{r_\text{max}}^ {r_\text{min} }dr ~ r^2 j_{l_\alpha}(k_\alpha r) \delta_\alpha(k_\alpha) \frac{1}{4\pi \Omega} \underbrace{ \int_{\theta_1}^{\theta_2 }\int_{\phi_1}^{\phi_2} d  \theta d \phi ~\sin(\theta)\sph{\alpha}{r}}_{a_{l_\alpha m_\alpha}} ~,
\end{align}
where $\Omega= a_{00}$. 

In practice we are interested in derivatives of observables with respect to the $\delta_\alpha$. This is accomplished by chain rule 
\begin{align}
\frac{\partial f}{\partial \delta_\alpha}= \frac{\partial f}{\partial \bar{\delta}} \frac{\partial \bar{\delta}}{\partial \delta_\alpha}~.
\end{align} 
The derivative is computed from Eqn.~\ref{bar_delta}
\begin{align}
\frac{\partial \bar{\delta} }{ \partial \delta_\alpha}=
\frac{3}{r_\text{max}^3 - r_\text{min}^3} \int_{r_\text{max}}^ {r_\text{min} }dr ~ r^2 j_{l_\alpha}(k_\alpha r)  \frac{1}{4\pi \Omega}  \int_{\theta_1}^{\theta_2 } \int_{\phi_1}^{\phi_2} d\theta d \phi ~\sin(\theta)\sph{\alpha}{r}~.
\end{align}


 
\section{Covariance of super modes}
In our basis the super survey mode is defined as 
\begin{align} 
\begin{split} 
\delta_\alpha(k_\alpha)& = \int \delta(\mathbf{r}) \jl{\alpha}(k_\alpha r) \sph{\alpha}{r} d^3 \mathbf{r}  \\
& = \int \dk \delta(\mathbf{k}) \int_0^{r_\text{max}} dr^2 \jl{\alpha}(k_\alpha r) \int d^2 \hat{r} e^{ikr \hat{k} \cdot \hat{r}} \sph{\alpha}{r} \\
& =4 \pi i^{l_\alpha}\int  \dk \delta(\mathbf{k})  \sph{\alpha}{k}  \int_0^{r_\text{max}} dr^2 \jl{\alpha}(k_\alpha r) \jl{\alpha}(kr)  ~,
\end{split} 
\end{align} 
where in the second equality $\delta(\mathbf{r})=(2\pi)^{-3} \int d^3 \mathbf{k} \exp(i \mathbf{k} \cdot \mathbf{r}) \delta(\mathbf{k})$ was used and in the third equality the following identity was used 
\begin{align} \int_{S^2} d^2 \hat{r} \sph{\alpha}{r} e^{i \mathbf{k} \cdot \mathbf{r}} = 4 \pi i^{l_\alpha} j_{l_\alpha}(k r)\sph{\alpha}{k}~.
\end{align}
Including a normalization\footnote{In our case $N_\alpha$ is 1.} $N_\alpha= \int_\Omega \sph{\alpha}{r} \sph{\alpha}{r} d^3 \mathbf{r} $ we have
\begin{align} 
\delta_\alpha= \frac{4 \pi i^{l_\alpha}}{N_\alpha} \int  \dk \delta(\mathbf{k})  \sph{\alpha}{k}  \int_0^{r_\text{max}} dr^2 \jl{\alpha}(k_\alpha r) \jl{\alpha}(kr)  ~.
\end{align} 
To get the covariance of the super survey field we take the ensemble average:
\begin{align} 
\begin{split} 
\langle \delta_\alpha \delta_\beta \rangle &= \frac{ (4 \pi)^2}{N_\alpha N_\beta} \int  \dkv{1} \dkv{2} \langle \delta(\mathbf{k}_1) \delta^*(\mathbf{k}_2) \rangle \sph{\alpha}{k_1} \sph{\alpha}{k_2} \\
& \;\;\; \times   \int_0^{r_\text{max}}  dr r^2 \jl{\alpha}(k_\alpha r) \jl{\alpha}( k_1 r) \int_0^{r_\text{max}}  dr r^2 \jl{\alpha}(k_\alpha r) \jl{\alpha}( k_2 r)  \\
&= \frac{ (4 \pi)^2}{N_\alpha N_\beta} \int \dk P(k) I_\alpha(k, r_\text{max}) \times I_\beta(k, r_\text{max})~,
\end{split} 
\end{align} 
where $(2 \pi)^3 \delta^3_\text{D}( \mathbf{k} + \mathbf{k}') P(k) =  \langle \delta(\mathbf{k}) \delta^*(\mathbf{k}') \rangle$ was used along with the definition 
\begin{align} 
\begin{split} 
I_\alpha(k,  r_\text{max}) & =  \int_0^{r_\text{max}}dr r^2 \jl{\alpha}(k_\alpha r) \jl{\alpha}( k r) \\
& = \frac{\pi}{2} \sqrt{ \frac{1}{k _ \alpha k}} \int_0^{r_\text{max}}dr r J_{l_\alpha+ 1/2}(k_\alpha r)  J_{l_\alpha+ 1/2}( k r) \\
& = \frac{\pi}{2} \frac{ r_\text{max}}{ \sqrt{ k _ \alpha k}}\frac{\left[J_{l_\alpha+ 1/2}( k  r_\text{max})J_{l_\alpha+ 1/2}'(k_\alpha  r_\text{max}) - J_{l_\alpha+ 1/2}(k_\alpha  r_\text{max})J_{l_\alpha+ 1/2}'(k  r_\text{max})\right]}{k^2 - k_\alpha^2}~.
\end{split} 
\end{align}

\section{The computation of basis objects}
The radial basis function should be zero at the sector boundary. Thus, fiven the maximum radius of the sector $r_\text{max}$ the super wave vector $k_\alpha$ is found by 
\begin{align}
 j_{l_\alpha}(k_\alpha r_\text{max})=0.
 \end{align}

\section{How to obtain derivatives by finite differences}
We need to calculate objects like $\partial \obs_I/ \partial \delta_\alpha$. This is done by applying the chain rule. 

\begin{align} 
\frac{\partial \obs_I}{ \partial \delta_\alpha} = \frac{\partial \obs_I}{ \partial \Theta^i} \frac{\partial \Theta^i}{\partial \bar{\delta} } \frac{\partial \bar{\delta}}{\delta_\alpha} ~. 
\end{align}
When the observable is to complicated that an analytic method for the derivative is not possible, or is too cumbersome to attempt, we can use a finite difference approach. We can take $\bar{\delta}=\pm \epsilon$. As long as $\epsilon$ is small we should be in the linear regime (which is the region we are limiting our analysis too). Then we can take 

\begin{align}
 \frac{\partial \obs_I}{ \partial \Theta^i} = \frac{ \obs_I(+ \epsilon) - \obs_I(- \epsilon)}{2\delta\Theta^I( \epsilon)} ~,
 \end{align} 
and in the above $  \obs_I(+ \epsilon)$ means that the observable is calculated in the separate universe approch, where $\bar{\delta}=\pm\epsilon$ corresponds to a change in cosmological parameters.  The figure below illustrates this approach. The point being, is that we can choose $\epsilon$ arbitrarily (as long as we are in the linear regime) and we will still get the correct derivative, because it is just a slope. 
 
 \begin{figure}[h]
  \caption{Finite difference approach.}
  \centering
    \includegraphics[width=\textwidth]{finite_diff_O.pdf}
\end{figure}

\end{document}  